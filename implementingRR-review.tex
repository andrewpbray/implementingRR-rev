\documentclass[12pt]{article}
\usepackage{url}

\setlength{\textwidth}{6.5in}
\setlength{\textheight}{9.0in}
\setlength{\oddsidemargin}{0in}
\setlength{\topmargin}{-0.5in}
\newenvironment{references}{
  \begin{center} \textsf{REFERENCES} \end{center}
  \begin{list}{}{\topsep=0pt\parsep=0pt\baselineskip=20pt
   \leftmargin=1.5em\itemindent=-\leftmargin}}
  {\end{list}}


\begin{document}

%Please complete with reviewer information
\noindent
Andrew \textsc{Bray}\\
University of Massachusetts, Amherst \\
Amherst, MA \\
760-519-5979 \\
andrew.bray@gmail.com\\
\medskip

%Please complete with book information
\noindent{\Large\sf Implementing Reproducible Research}
\begin{quote}
Victoria \textsc{Stodden},
Friedrich \textsc{Leisch},
Roger D. \textsc{Peng}.

Publisher Address: CRC Press, 2013. ISBN 978-1-4665-6159-5. 
%Replace with front and main body pages.
xix+428 pp. 
\end{quote}

\thispagestyle{empty}
\raggedright\baselineskip=18pt\parindent=2em\parskip=5pt

%%%%% ENTER THE BODY OF THE BOOK REVIEW BELOW %%%%%
Reproducibility is defined as, ``the calculation of quantitiative scientific results by
independent scientists using the original datasets and methods'' (p.\ vii). The challenge
of reproducibility in the computational era is being confronted across the 
sciences, with each field developing its own tools and best practices.   This
book is an important step in bringing together a broad group of scientists to 
share what has been learned.

In the preface, the editors specify that their objective is not to convince the
reader \emph{why} reproducibility is important, but rather \emph{how} it can 
be achieved. They organize the book into three parts, Tools, Practices and 
Guidelines, and Platforms, that mirror the three directions of research in 
reproducibility. One contributor then usefully divides tools for reproducible research into three 
general categories: tools for literate programming (Ch. 1: knitr in R), workflow management 
systems (Ch. 2: VisTrails), and tools for environment capture (Ch. 3: Sumatra, Ch. 4: CDE). 
I was most familiar with knitr, but I still learned alot from the discussion
 of the design elements that specifically enable reproducibility, including chunk 
re-use and conditional evaluation. The remaining tools were new to me and address
the added challenges when faced with a more complex workflow and software that
involves many dependencies. Note that these chapters are not full
tutorials; the reader will not emerge as a well-versed user, but hopefully with a 
sense of which tools they would like to invest time in learning.

The remainder of the book is a mix of more general and technical pieces,
some of which are quite domain specific.  Chapters 5, 7, and 13 discuss methods
from physical, biological science, and machine learning, respectively, though
their approach to computation may be of interest to those in other fields. Chapters 
6, 9, and 15 discuss arguments and techniques for practicing open science, an
important consideration if the goal is to allow anyone to reproduce a study
without barriers.  Chapter 8 provides the prospective from within industry of a 
large-scale data anlysis project. Chapter 10 describes how cloud-based virtual 
machines (VM) can enable anyone to recreate a computational experience in an 
environment identical to that of the original authors. Chapter 12 reviews the
tenets of traditional intellectual property law and the discusses the difficulty
 this poses for reproducibility. Chapter 14 introduces the reader to runmycode.org,
a user-friendly web-based platform that allows one to tweak-the-knobs, so to say,
on an analysis.

This book should have incredibly broad appeal; the more general chapters 
from this book should be of interest to any empirical scientist.  Most contributors
assume familiarity with scientific computing. I can imagine this book serving as
a guide to improved reproducibility within a research lab or as terrific  
material to kickstart discussion for a university working group on reproducibility.

As a cover-to-cover read, the book can be repetitive (literate programming, version
control, and topics in open science are covered in multiple chapters).  It can also be
disorganized, with sections varying greatly in level of technical background
(e.g. Chapter 4 assumes familiarity with Linux while Chapter 8 reverts to a primer on
the basics of computing). In the preface, the editors describe six chapters to appear
in the Tools section, yet there are only four to be found. Additionally, some readers might be 
surprised to find that, for a book published as part of \emph{The R Series}, 
Python gets as much coverage as R, and much of the book is language-agnostic.
Considering the many arguments that this book makes for open science, it was
reassuring to find that full pdfs of every chapter (including the missing one by 
Malik et al.) can be downloaded for free through the Open Science Framework 
(implementingrr.org). One can't help but feel this to be the more natural format
 for this material: dynamic, modular, and with no barriers to access.

Despite some weaknesses of the book format, \emph{Implementing Reproducible
Research} still introduces some extremely useful tools and practices from leaders in 
the field.  On top of that, it also contains exciting visions for the future of scientific research.
One is the idea of learning from research \emph{processes} instead of just the 
end result, which becomes feasible when work is reproducible and transparent.
Part of this will be shift away from publishing static papers to publishing full research
environments.  Another is the promise of collaborative science, from the distributed
replication studies of The Reproducibility Project (Ch. 11) to the reevaluation of traditional 
peer review practices in light of the paradigm epitomized by GitHub's pull-request.

In sum: do not buy this book. Do download pdfs of the chapters of interest to you
and distribute them far and wide. This is vitally important stuff.


%%%%% END OF REVIEW 
%Complete with reviewer information.

\begin{flushright}\def\baselinestretch{1}
Andrew \textsc{Bray}\\
\emph{University of Massachusetts, Amherst}

\end{flushright}


%------ IF THERE ARE NO REFERENCES, UNCOMMENT THE NEXT LINE ------
%\begin{references}
%
%\end{references}

\end{document}



