\documentclass[12pt]{article}
\usepackage{url}

\setlength{\textwidth}{6.5in}
\setlength{\textheight}{9.0in}
\setlength{\oddsidemargin}{0in}
\setlength{\topmargin}{-0.5in}
\newenvironment{references}{
  \begin{center} \textsf{REFERENCES} \end{center}
  \begin{list}{}{\topsep=0pt\parsep=0pt\baselineskip=20pt
   \leftmargin=1.5em\itemindent=-\leftmargin}}
  {\end{list}}


\begin{document}

%Please complete with reviewer information
\noindent
Andrew \textsc{Bray}\\
University of Massachusetts, Amherst \\
Amherst, MA \\
760-519-5979 \\
andrew.bray@gmail.com\\
\medskip

%Please complete with book information
\noindent{\Large\sf Implementing Reproducible Research}
\begin{quote}
Victoria \textsc{Stodden},
Friedrich \textsc{Leisch},
Roger D. \textsc{Peng}.

Publisher Address: CRC Press, 2013. ISBN 978-1-4665-6159-5. 
%Replace with front and main body pages.
xix+428 pp. 
\end{quote}

\thispagestyle{empty}
\raggedright\baselineskip=18pt\parindent=2em\parskip=5pt

%%%%% ENTER THE BODY OF THE BOOK REVIEW BELOW %%%%%
The challenge of reproducing scientific results in the computational era is 
being dealt with across the sciences, with each field developing its own
 tools and best practices.  This book is an important step in bringing together
 a broad group of scientists to share what they have learned and current best
practices.  In the preface to his book, the editors clarify that their objective
is not to convince the read \emph{why} reproducibility is important, but rather
\emph{how} it can be achieved. Reproducibility is defined as the calculation of
quantitiative scientific results by independent scientists using the original
datasets and methods.

The chapters from the various contributors are organized into three sections -
 Tools, Practices and Guidelines, and Platforms - that mirror the three strains 
of research in reproducibility. Yihui Xie leads off with a chapter on the knitr
package for R.  Although I use the package regularly, I benefitted greatly from
discussion of the design elements that specifically enable reproducibility,
including chunk re-use and conditional evaluation.

As an user of R and the knitr package, I really enjoyed getting to learn 
tools and methods of the Python-verse, including IPython and org-mode.

The most frustrating part of reading this book is the slight disconnect between
the recommendations of the authors and final product of the book

Several of the chapters discuss how reproducibility can be facilitated by working
with open data, open source software, open access journals.  I was left
wondering why the authors chose to publish their work as a printed book costing
\$60 (it appears that none of the authors used a SPARC license to allow for 
access to chapters on their personal websites, as is recommended in chapter ZZQ).
As the 

Given the broad coverage of fields and range in technical detail, it is
difficult to know who the target audience for this book is.

Although it was not noted anywhere in the text, full pdfs of each chapter
can be downloaded for free through the Open Science Framework (implementingrr.org).
This was a relief for me: several of the 

In sum: do not buy this book. Do read the pdfs of the chapters of interest to you
and distribute them wide and far. This is vitally important stuff.



























Circular data refers to data that may be thought of as points on the
unit circle, such as wind direction, or time of day.  As the authors
note, there are not many books available on the topic, the most recent
being Mardia and Jupp (1999) and Jammalamadaka and SenGupta (2001),
which are more theory-oriented texts.  The authors state that they
aimed to produce a short, modern, computer based introduction to the
analysis of circular data which would be useful to both scientists and
statisticians.  They make extensive use of the R {\em circular}
package, and include some code of their own.  I came to this book as a
long time R user, but having no experience analyzing circular data,
so I was curious to see their approach.

The first chapter gives a brief introduction to circular statistics
and R, sensibly directing those who don't know R to consult other
resources, including internet sources such as the nearest CRAN mirror.
It also introduces the R {\em circular} package and makes note of some
of the default choices that affect circular data.  The authors have
established a website for code and data used in the book which also
includes an R workspace containing those items.  The text has a
straightforward organization.  Chapter 2 covers graphical methods for
circular data, and Chapter 3 descriptive statistics.  The authors warn
the reader of the existence alternative definitions of variance, and
helpfully suggest clues that one might use to guess which was used in
a published paper that didn't explicitly state which definition was
used.  Chapter 4 presents definitions of moments and densities for
circular data.  Chapter 5 covers some basic elements of inference:
tests of uniformity and symmetry, bootstrapping, and testing a null
hypothesis about the mean.  Chapter 6 covers maximum likelihood
estimation for the unimodal distributions presented in Chapter 4.
Chapter 7 covers the comparison of two samples, and Chapter 8 deals
with regression models.  Overall I found the presentation clear, if
rather brief.  I suspect that some sections would be challenging for
scientists without a strong background in mathematics.  In particular
the definition of the trigonometric moments is completely abstract,
with no examples given to help the neophyte.  This brevity should not
be a problem for the scientist who cares only to acquire the R tools
to analyze circular data, but might be a challenge for those who wish
to achieve a deeper understanding.  The authors do refer readers to
other texts on circular data which cover the theory in greater detail.

As a long time R user, I had a few quibbles.  In Chapter 1 the authors
give instructions for installing the circular package that make use of
a menu interface instead of the simple and direct
{\tt install.packages("circular")}, and they give examples using the
dangerous shorthand {\tt 'T'} for {\tt 'TRUE'}, though the code in later chapters
consistently uses {\tt 'TRUE'}.  The code presented does not conform to any
standard style guidelines for R code, for example there is no use of
indentation, and often two commands are entered on a single line,
separated by a semi-colon.  These are minor irritations rather than
serious failings.

I also question the analytical advice at the beginning of Chapter 5
where the authors suggest that one start with a test of uniformity,
and if that null hypothesis is rejected, test for symmetry.  They then
suggest fitting the Jones-Pewsy family (no relation) if the null hypothesis of
symmetry is not rejected, and something like an inverse Batschelet
distribution if it is.  This seems akin to testing for normality
before fitting a normal distribution.  As the authors later note, one
can fit the inverse Batschelet and test null hypotheses about
parameters corresponding to symmetry or reduction to the von Mises
distribution via likelihood ratio tests. The authors also provide
functions for quantile-quantile plots.  In my opinion, fitting a
plausible distribution followed by checking diagnostic plots should be
the canonical practice, rather than a sequence of preliminary tests or
a formal goodness-of-fit test.

The strong points of this text: how to use the R {\em circular}
package, with datasets and examples to illustrate the methods.  The
authors have created a website (\url{http://circstatinr.st-andrews.ac.uk})
where one may find R code and datasets, as well as an R
workspace containing those items.  They have provided R functions for
fitting recently developed distributions and other modern methods such
as bootstrapping.  One who already knows R can quickly get up to
speed.  I think the text will be useful for self-study by scientists
and statisticians, and potentially useful in some applied courses or
as a supplement to a more theoretical course.  There is much it doesn't
cover in any depth, including mixture distributions and Bayesian inference,
but perhaps that is too much to ask of a concise introduction.



%%%%% END OF REVIEW 
%Complete with reviewer information.

\begin{flushright}\def\baselinestretch{1}
Andrew \textsc{Bray}\\
\emph{University of Massachusetts, Amherst}

\end{flushright}


%------ IF THERE ARE NO REFERENCES, UNCOMMENT THE NEXT LINE ------
%\begin{references}
%
%\end{references}

\end{document}



